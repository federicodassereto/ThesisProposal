%%%%%%%%%%%%%%%%%%%%%%%%%% Proposal of the first year %%%%%%%%%%%%%
\documentclass[a4paper,11pt, english]{article}

%%%%%%%%%%%% Packages
\usepackage[table]{xcolor}         			% coloured text
\usepackage{smartdiagram}
\usepackage{adjustbox}
\usepackage{changepage}
\usepackage{standalone}
\usepackage[utf8]{inputenc}
\usepackage{hyperref}
\usepackage{lipsum,calc}
\usepackage{array,longtable}
\usepackage{tabularx} 
\usepackage{collcell}
\usepackage{pgfplots}
\usepackage{graphicx}
\usepackage[all]{xy}         			% diagrams
\usepackage{amssymb}				% \mathbb and others
\usepackage{amsthm}				% theorem styles
\usepackage{amsmath}				% Declair math operator
\usepackage{enumerate}       			% many possible numerations
\usepackage{manfnt,xspace}			%% added package
\usepackage[a4paper,top=3cm, bottom=3cm, left=3cm, right=3cm]{geometry}
\usepackage{pifont}% http://ctan.org/pkg/pifont
%\usepackage{natbib}
\usepackage[backend=bibtex,style=numeric,citestyle=numeric]{biblatex}
\addbibresource{proposal_correct}
                        

\newcommand{\cmark}{\ding{51}}                    
                        
\definecolor{lightgreen}{HTML}{CCFF66}
\definecolor{green}{HTML}{66FF66}
\definecolor{lightyellow}{HTML}{FFFF66}
\definecolor{orange}{HTML}{FF9900}
\definecolor{red}{HTML}{FF3333}
\definecolor{cyan}{HTML}{00CCFF}
\definecolor{blue}{HTML}{0000FF}
\definecolor{blue-violet}{rgb}{0.54, 0.17, 0.89}
                        

%%%%% Font Macros
\usepackage{dsfont}
\usepackage[T1]{fontenc}
\def\lqq{{\fontencoding{T1}\selectfont\guillemotleft}}
\def\rqq{{\fontencoding{T1}\selectfont\guillemotright}}
\DeclareFontFamily{OT1}{pzc}{}
\DeclareFontShape{OT1}{pzc}{m}{it}{<->s*[1.10]pzcmi7t}{}
\DeclareMathAlphabet{\mathpzc}{OT1}{pzc}{m}{it}


\pgfplotsset{compat=1.9}

%%%%%%%%%%%%%%%%%%%%%%% title

\title{\vspace{0cm}
\hrule
\vspace{1cm}
\centerline{\LARGE{Thesis Proposal:}}
\vspace{0.5cm}
\centerline{\LARGE {\bf The Role of Embeddings in Data-Driven Augmentation}}
%\centerline{\LARGE{\bf }}
}

%\title{\begin{Large}Research Proposal\end{Large}}
\author{Federico Dassereto}
\date{\today}

%%%%%%%%%%%%%%%%%%%%%%%%%%%%%%%%%%%%%%%%% DOCUMENT %%%%%%%%%%%%%%%%%%%%%%%%%%%%%%%%%%%%%%%%%%%

\begin{document}
\maketitle
\hrule 
\vspace{1cm}

\begin{abstract}
    A-o meu neuo gh'é neue nae neue: a ciù neua de neue nae neue a n'eu anâ.
\end{abstract}

\section{Motivation and Description}\label{motivations}
In data science, it is increasingly the case that the main challenge is not in integrating known data, rather it is in finding the right data to solve a given data science problem. Today, data is a mass (uncountable) noun like dust, and data surrounds us like dust, even lovely structured data. Data is so cheap and easy to obtain that it is no longer important to always get the integration right and integrations are not static things. Data integration research has embraced and prospered by using approximation and machine learning. The uncontrolled nature of data manifests in large repositories of data (data lakes), in which both structured and unstructured data are stored. The peculiarity of data lakes lies in the fact that there is uncertainty about the presence of metadata describing the data themself. Furthermore, it is common the situation in which there is a lack of schemas, making traditional database approaches to integrating or querying data difficult to pursuit or even infeasible. Along with the uncertainty regarding the quality of the data, data Volume makes it infeasible the traditional human-in-the-loop framework, since hand labeling or manual rating of very large amounts of data is extremely expensive.

\section{Reference Area and Relevance of Goals}\label{reference}
To the best of our knowledge, no automatic table augmentation framework adapt to work at Internet Scale exists. We believe that such a framework would be very relevant to exploit the large amount of unstructured data available on the web and on data lakes, allowing not only to improve machine learning models but also to retrieve semantic information from a variety of schemaless tables.
The proposal lies in the reference area "Data Augmentation to improve Machine Learning Tasks", intending to provide to a data scientist an augmented set of data which enables her to improve model performances. The approach we propose differ from other existing solutions in the following aspects:
\begin{itemize}
    \item Existing approaches mainly focus on predicting the usefulness of a join under the relational hypothesis, i.e., the schema information are known. To this end, we plan to develop an approach that is agnostic to the knowledge of the schemas.
    \item Existing approaches that try to augment data against a repository perform very expensive joins, and are evaluated concerning repositories of limited size; other existing approaches on repositories run features selection algorithms on a very large matrix derived by the join of all the joinable tables. To this end, we plan to develop an approach that does not materialize any kind of join and is scalable to hundreds of thousands of tables in a repository.
    \item Existing approaches on single tables identify the most relevant features in the table, according to a specific target. We plan to overcome the single table assumption by indexing all the tables in a repository and identifying the most relevant features, and the relative tables, that improve the machine learning model performances.
    \item Other approaches simply return the set of joinable tables, usually with a threshold to allow for relaxed  (imperfect) join, without any ranking. We plan to rank the returned tables according to the improvement that each of them will guarantee to the model.
\end{itemize}    

\section{State of the Art}\label{related}

\section{Goals and Preliminaries}\label{goals}

\section{Research Plan}\label{researchplan}

We aim at organizing our research in the three years according to eight activities, listed in Table 1. Each activity corresponds to a thesis objective or to the thesis writing. In particular, Table \ref{table1}, for each activity, points out each sub-objective and the time units dedicated to it, while Table \ref{table2} presents their schedule in the three years.

% 1 Analysis of the state of the art on open data, sketching and approximation of information theory measures
% 2 Definition of an index for tables augmentation in data lakes based on information-theoretic measures
% 3 Analysis of the state of the art on functional dependency discovery algorithms

% 4 Analysis of kwnoledge bases representations and augmentations
% 5 Definition of a framework to include embeddings of knowledge bases in tables augmentation

% 6 Definition of an algorithm to embed knowledge bases
% 7 Evaluation of the performances of the knwoledge bases embedding algorithm in answering queries

% 8 PhD thesis writing

\begin{table}[h!]\footnotesize
    \centering
    
    \caption{Workplan\label{table1}}
    \begin{tabular}{|c|p{9cm}|c|c|}
    \hline
    \textbf{Activity} & \textbf{Description} & \textbf{Months} & \textbf{Colour}\\ \hline
    1 & Analysis of the state of the art on open data, sketching and approximation of information theory measures & 1st Year & \cellcolor{lightgreen} \\\hline
    2 & Definition of an index for tables augmentation in data lakes based on information-theoretic measures& 1st Year & \cellcolor{green} \\\hline
    3 & Analysis of the state of the art on functional dependency discovery algorithms& 6 & \cellcolor{lightyellow} \\\hline
    4 & Analysis of kwnoledge bases representations and augmentations& 6 & \cellcolor{orange} \\\hline
    5 & Definition of a framework to include embeddings of knowledge bases in tables augmentation& 6 & \cellcolor{red} \\\hline
    6 & Definition of an algorithm to embed knowledge bases& 6 & \cellcolor{cyan} \\\hline
    7 & Evaluation of the performances of the knwoledge bases embedding algorithm in answering queries& 6 & \cellcolor{blue} \\\hline
    8 & PhD thesis writing& 4 & \cellcolor{blue-violet} \\\hline
    \end{tabular}
    \end{table}
    
    \bigbreak
    \vspace{-2em}
    
    \newcolumntype{C}[1]{>{\centering\arraybackslash}m{#1}}
    %\begin{center}
    \begin{table}[h!]\footnotesize

    \caption{Workplan\label{table2}}
    \centering
    %\caption{{\bf A} = period abroad\label{table2}}
    
    \begin{adjustwidth}{-1.2cm}{}
    \begin{tabular}{|c|*{36}{C{.01cm}|}}%{|c|*{36}{p{.01cm}|}}
    \hline
    \textbf{Activity} & \multicolumn{12}{c|}{Year1} & \multicolumn{12}{c|}{Year2} & \multicolumn{12}{c|}{Year3} \\\hline
    %\cellcolor{orange}
                   1  &&&&&&&&&&&&&&&&&&&&&&&&&&&&&&&&&&&&\\\hline
                   2  &&&&&&&&&&&&&&&&&&&&&&&&&&&&&&&&&&&&\\\hline
                   3  &&&&&&&&&&&&&&&&&&&&&&&&&&&&&&&&&&&&\\\hline
                   4  &&&&&&&&&&&&&&&&&&&&&&&&&&&&&&&&&&&&\\\hline
                   4  &&&&&&&&&&&&&&&&&&&&&&&&&&&&&&&&&&&&\\\hline
                   4  &&&&&&&&&&&&&&&&&&&&&&&&&&&&&&&&&&&&\\\hline
                   7  &&&&&&&&&&&&&&&&&&&&&&&&&&&&&&&&&&&&\\\hline
                   8  &&&&&&&&&&&&&&&&&&&&&&&&&&&&&&&&&&&&\\\hline
    \end{tabular}
    \end{adjustwidth}
    \end{table}

\section*{Acknowledgements}{So long, and thanks for all the fish.}



%\nocite{*}
\printbibliography

\end{document}