\section{Research Plan}\label{researchplan}
We distinguished our objectives in different classes, i.e., the objectives of Section \ref{sub_goals} that are supposed to be completed in a short time and objectives which require more work to be completed. This distinction is reported in Table \ref{tab:completed}.

We aim at organizing our research in the three years according to eight activities, listed in Table \ref{table1}. Each activity corresponds to a thesis objective or to the thesis writing. In particular, Table \ref{table1}, for each activity, points out each sub-objective and the time units dedicated to it, while Table \ref{table2} presents their schedule in the three years.

% 1 Analysis of the state of the art on open data, sketching and approximation of information theory measures
% 2 Definition of an index for tables augmentation in data lakes based on information-theoretic measures
% 3 Analysis of the state of the art on functional dependency discovery algorithms

% 4 Analysis of knowledge bases representations and augmentations
% 5 Definition of a framework to include embeddings of knowledge bases in tables augmentation

% 6 Definition of an algorithm to embed knowledge bases
% 7 Evaluation of the performances of the knowledge bases embedding algorithm in answering queries

% 8 PhD thesis writing

% \begin{table}[h]
%     \centering    
%     \begin{tabular}{|c|p{5cm}|}
%     \hline
%     \textbf{Objective} & \textbf{Long/Short Term Task} \\ \hline
%     1.1 & Short Term Task\\ \hline
%     1.2 & Short Term Task\\ \hline
%     2.1 & Short Term Task\\ \hline
%     2.2 & Long Term Task \\ \hline
%     3 & Long Term Task \\ \hline
%     4   & Always in Progress \\ \hline
%     5   & Long Term Task \\ \hline
%     \end{tabular}
%     \caption{Long/Short Term feasibility of each objectives presented in Section \ref{goals}. \label{tab:completed}}
%  \end{table}

\begin{table}[h]
    \centering    
    \begin{tabular}{|c|p{5cm}|}
    \hline
    \textbf{Objective} & \textbf{Long/Short Term Task} \\ \hline
    1.1, 1.2, 1.3 & Short Term Task\\ \hline
    2.1 & Short Term Task\\ \hline
    2.2, 2.3 & Long Term Task \\ \hline
    3.1, 3.2 & Long Term Task \\ \hline
    3.3   & Always in Progress \\ \hline
    PhD Thesis Writing & Long Term Task \\ \hline
    \end{tabular}
    \caption{Long/Short Term feasibility of every objective presented in Section \ref{goals}. \label{tab:completed}}
 \end{table}


% \begin{table}[h!]\footnotesize
%     \centering
    
%     \caption{Activities\label{table1}}
%     \begin{tabular}{|c|p{9cm}|c|c|}
%     \hline
%     \textbf{Activity} & \textbf{Description} & \textbf{Months} & \textbf{Colour}\\ \hline
%     1 & Analysis of the state of the art on open data, sketching and approximation of information theory measures & 1st Year & \cellcolor{lightgreen} \\\hline
%     2 & Definition of an index for tables augmentation in data lakes based on information-theoretic measures& 1st Year & \cellcolor{green} \\\hline
%     3 & Analysis of the state of the art approaches for knowledge bases construction, representation and augmentation& 8 & \cellcolor{lightyellow} \\\hline
%     4 & Definition of a framework that exploits knowledge bases and possibility of including embedding to augment and Comparison & 8 & \cellcolor{orange} \\\hline
%     5 & Definition of an embedding algorithm for knowledge bases & 6 & \cellcolor{red} \\\hline
%     6 & Continuative analysis of the state of the art approaches embeddings works & 16 & \cellcolor{cyan} \\\hline
%     7 & Experimental Evaluation& 10 & \cellcolor{blue} \\\hline
%     8 & PhD thesis writing& 4 & \cellcolor{blue-violet} \\\hline
%     \end{tabular}
% \end{table}

\begin{table}[h!]\footnotesize
    \centering
    
    \caption{Activities\label{table1}}
    \begin{tabular}{|c|p{9cm}|c|c|}
        \hline
        \textbf{Activity} & \textbf{Description} & \textbf{Months} & \textbf{Colour}\\ \hline
        1 & Analysis of the state of the art on open data, sketching and approximation of information theory measures & 1st Year & \cellcolor{lightgreen} \\\hline
        2 & Definition of an index for tables augmentation in data lakes based on information-theoretic measures& 1st Year & \cellcolor{green} \\\hline
        3 & Analysis of the state of the art approaches for knowledge bases construction, representation and augmentation & 8 & \cellcolor{lightyellow} \\\hline
        4 & Definition of a framework that exploits embedded knowledge bases to augment & 6 & \cellcolor{orange} \\\hline
        5 & Definition of an embedding algorithm for knowledge bases & 6 & \cellcolor{red} \\\hline
        6 & Continuative analysis of the state of the art approaches embeddings works & 16 & \cellcolor{cyan} \\\hline
        7 & Experimental Evaluation & 10 & \cellcolor{blue} \\\hline
        8 & PhD thesis writing & 4 & \cellcolor{blue-violet} \\\hline
    \end{tabular}
\end{table}
    
    \bigbreak
    \vspace{-2em}
    
    \newcolumntype{C}[1]{>{\centering\arraybackslash}m{#1}}
    %\begin{center}
    \begin{table}[h!]\footnotesize

    \caption{Workplan\label{table2}}
    \centering
    %\caption{{\bf A} = period abroad\label{table2}}
    
    \begin{adjustwidth}{-1.2cm}{}
    \begin{tabular}{|c|*{36}{C{.01cm}|}}%{|c|*{36}{p{.01cm}|}}
    \hline
    \textbf{Activity} & \multicolumn{12}{c|}{Year1} & \multicolumn{12}{c|}{Year2} & \multicolumn{12}{c|}{Year3} \\\hline
    %\cellcolor{orange}
    1  &\cellcolor{lightgreen}&\cellcolor{lightgreen}&\cellcolor{lightgreen}&\cellcolor{lightgreen}&\cellcolor{lightgreen}&\cellcolor{lightgreen}&\cellcolor{lightgreen}&\cellcolor{lightgreen}&\cellcolor{lightgreen}&\cellcolor{lightgreen}&\cellcolor{lightgreen}&\cellcolor{lightgreen}&&&&&&&&&&&&&&&&&&&&&&&&\\\hline
    2  &&&&&&\cellcolor{green}&\cellcolor{green}&\cellcolor{green}&\cellcolor{green}&\cellcolor{green}&\cellcolor{green}&\cellcolor{green}&&&&&&&&&&&&&&&&&&&&&&&&\\\hline
    3  &&&&&&&&&&&&&\cellcolor{lightyellow}&\cellcolor{lightyellow}&\cellcolor{lightyellow}&\cellcolor{lightyellow}&\cellcolor{lightyellow}&&&&&&&&\cellcolor{lightyellow}&\cellcolor{lightyellow}&\cellcolor{lightyellow}&&&&&&&&&\\\hline
    4  &&&&&&&&&&&&&&&&&\cellcolor{orange}&\cellcolor{orange}&\cellcolor{orange}&\cellcolor{orange}&\cellcolor{orange}&\cellcolor{orange}&&&&&&&&&&&&&&\\\hline
    5  &&&&&&&&&&&&&&&&&&&&&&&&&&&\cellcolor{red}&\cellcolor{red}&\cellcolor{red}&\cellcolor{red}&\cellcolor{red}&\cellcolor{red}&&&&\\\hline
    6  &\cellcolor{cyan}&\cellcolor{cyan}&\cellcolor{cyan}&\cellcolor{cyan}&&&&&\cellcolor{cyan}&\cellcolor{cyan}&&&&\cellcolor{cyan}&\cellcolor{cyan}&\cellcolor{cyan}&&&&\cellcolor{cyan}&\cellcolor{cyan}&&&&\cellcolor{cyan}&\cellcolor{cyan}&\cellcolor{cyan}&&\cellcolor{cyan}&\cellcolor{cyan}&&&&&&\\\hline
    7  &&&&&&&&&&&\cellcolor{blue}&\cellcolor{blue}&&&&&&&&&\cellcolor{blue}&\cellcolor{blue}&\cellcolor{blue}&\cellcolor{blue}&&&&&&&\cellcolor{blue}&\cellcolor{blue}&\cellcolor{blue}&\cellcolor{blue}&&\\\hline
    8  &&&&&&&&&&&&&&&&&&&&&&&&&&&&&&&&&\cellcolor{blue-violet}&\cellcolor{blue-violet}&\cellcolor{blue-violet}&\cellcolor{blue-violet}\\\hline
    \end{tabular}
    \end{adjustwidth}
    \end{table}