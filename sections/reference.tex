\section{Reference Area and Relevance of Goals}\label{reference}
The current proposal lies in the reference area "Data Augmentation to improve Machine Learning Tasks", with the aim of providing to a data scientist an augmented set of data which enables her to improve her model performances. The approach we propose differ from other existing solutions in the following aspects:
\begin{itemize}
    \item Existing approaches mainly focus on predicting the uselfulness of a join under the relational hypothesis, i.e., the schema informations are known. To this end, we plan to develop an approach that is agnostic to the knowledge of the schemas;
    \item Existing approaches which try to augment data against a repository perform very expensive joins, and are evaluated with respect to repositories of limited size; other existing approaches on repositories run features selection algorithms on a very large matrix derived by the join of all the joinable tables. To this end, we plan to develop an approach that does not materialize any kind of join and is scalable to hundreds of thousands tables in a repository;
    \item Existing works on single tables identify the most relevant features in the table, according to a specific target. We plan to overcome the single table assumption by indexing all the tables in a repository and identifying the most relevant features, and the relative tables, that improve the machine learning model performances;
    \item Other approaches simply return the set of joinable tables, usually with a treshold to allow for relaxed  (imperfect) join, without any ranking. We plan to rank the returned tables according to the improvement that each of them will guarantee to the model.
\end{itemize}    